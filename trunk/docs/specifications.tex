\documentclass[10pt,fullpage]{article}

\usepackage{hyperref}                 % For creating hyperlinks in cross references
%\usepackage{ieee8}
\usepackage{times}

\topmargin -1.5cm \oddsidemargin -0.04cm \evensidemargin -0.04cm
\textwidth 16.00cm \textheight 23.50cm
\parskip 7.2pt
\parindent 0.25in

\makeindex

\title{ Humane Assembly Language Tools \\ Software Specification}


\author{Matthew Bennett, Michael Erwin \\
{{\em School of Computing. University of Southern Mississippi.
Hattiesburg, MS 39406 } } \\ {\em matthew.bennett@usm.edu,
michael.erwin@usm.edu } }

\date{ }

\begin{document}
\maketitle

\section*{Statement of Purpose}

The Humane Assembly Language Tools (HALT) project seeks to provide a
simple and efficient development and execution environment for the
transition from higher level languages.

HALT is a toolkit for user-friendly development and inspection of
Motorola 68000 assembly language. The philosophy behind HALT is to
make assembly language as accessible as possible to a broad audience
of programmers. That philosophy is realized through a simple,
colorful human interface, connecting the user to advanced tools such
as a powerful lexxer/parser, a bare M68000 machine language
interpreter, an M68000 assembler and translator, and various
debugging and execution tools.

HALT provides a simplified run-time visualization for the internal
working environment of a virtual M68000 machine. The visualization
updates the state of the stack as new code is typed into the working
project. The simulator is also a visualization environment for
program execution, and displays the contents of registers and memory
as it occurs once the assembly instructions have been successfully
translated and interpreted. All this is done at development time,
within one simple and easy-to-use framework, so the developer�s time
to product is minimized. HALT also functions well in a teaching
environment, as it follows the KISS principle: Make everything as
simple as possible, but no simpler. The bright and simple display of
information make the user interface a fun and powerful way to learn
and develop Motorola 68000 assembly language code.

HALT is also a tool for developers. HALT produces only machine
instructions which are in a strict subset of the Motorola 68000
machine instruction set. Therefore, any program in HALT should also
run on any machine that implements the basic 68000 instruction set.

This document provides a guideline for the software development
process. The content should reflect those points expressed in the
HALT philosophy\cite{phil}. All specifications outlined herein are
mutable if they do not reflect the later choices of the developers.

\section*{Platform}

To provide the widest accessability to academia, HALT will be
cross-platform. Target platforms are Microsoft Windows (2000, XP,
Vista, and NT), Linux (xorg), and OS X. Portable source code will be
available so that HALT may be transferred to future desirable
platforms with minimal effort.

\subsection*{Implementation Language}

All code associated with HALT must be written in C++. Libraries used
will be cmath, STL, OpenGL, and GLUT. Any other libraries must be
dynamically compiled, and must run on Windows, Linux, and Windows.
All code associated with HALT must be written to compile and run on
Windows, Linux, and Apple OS X.

\subsection*{Development Process}

Fully collaborative development will be accomplished using a
non-locking version control (merging) version control system,
subversion. It will also allow easy maintenance of code branches,
and to ``rewind'' mal-appropriate code changes. The code for all
versions after the initial release are browseable with change logs
and blame tags.

\subsection*{Program Design}

HALT will consist of several object modules, each with its own
responsibility. Since HALT is an integrated development environment
consisting of an editor, simulator, and machine inspector,
functionality will be tightly coupled within objects. The User's
Guide \cite{user} provides as an Appendix the class dependency and
class inheritance diagrams.

\subsection*{Documentation}

Good documentation is extremely important, as the product will be
used primarily within academia. User support, bug tracking, and
feature requests are also imperative.

All documents, including this one, will be written in the \LaTeX
typesetting system, and made available in both .PDF and bound hard
copy. This is so it can be treated as source code in the subversion
version control system, as well as providing for later publication
of reference materials.

Two web sites will be provided. A General web site will provide
news, release versions, documentation, and contact information, and
can be found at \cite{home}. A more extensive development site will
provide bug tracking, feature requests, a discussion forum, version
control with code rollback, multi-lingual Wiki, mailing lists, task
assignment, and other features. It is located at \cite{berlios}.

\subsection*{Assembly Language Features}

The language shall be a subset of Motorola 68000 Assembly. Following
is a list of commands that must be supported before HALT 1.0.

\noindent Instructions: {\bf \small STOP, MOV, ADD, SUB, MUL, DIV,
AND, OR, NOT, EOR,
LEA, BRA, BLE, BGT, BLE, BGE, BNE, CLR, NEG, NOP, BSR}.\\
Addressing modes: {\bf \small Data Register Direct, Address Register
Direct, Address Register Indirect, Address Register Indirect with
Pre-Decrement, Address Register Indirect with Post-Increment,
Symbolic Address Indirect, Literal}.

All assembly language instructions must behave in the simulator and
virtual machine as they are defined in the Motorola 68000
Programmer's Reference\cite{moto}. Language features which are not
listed here are candidates for inclusion in HALT's subset of
Motorola 68000 assembly, but are not required. All language choices
must reflect the HALT design philosophy\cite{phil}.

\subsection*{Added Value Features}

Some higher-level features such as array declaration, and indirect
indexing may be provided, even though they are not part of the
original Motorola 68000 instruction set\cite{moto}. These features
must be implemented in such a way that the machine language produced
by HALT's assembler is still a strict subset of the Motorola 68000
machine language. In this way, programs developed with HALT are
assured to run on any 68k-based hardware.
\newpage
\subsection*{Tentative Acknowledgements}

\begin{itemize}
\item {\bf BerliOS} for hosting the project web site and
providing project management support.
\item {\bf NASA} for providing funds for IDCC'05-06.
\item {\bf USM School of Computing} for hosting IDCC'05-06.

\end{itemize}

\begin{thebibliography}{4}

\bibitem{moto} Motorola Corporation. {\bf Motorola M68000 Programmer's Reference
Manual}. 1992. \\
Available at \href{www.freescale.com}{www.freescale.com} or with
request from Motorola Corporation.

\bibitem{user} Matthew Bennett and Michael Erwin. {\bf HALT User's Guide}. 2006. \\ Available at
\href{http://halttool.berlios.de}{http://halttool.berlios.de} or
printed by request.

\bibitem{phil} Michael Erwin. {\bf HALT Design Philosophy}. 2006. \\ Available at
\href{http://halttool.berlios.de}{http://halttool.berlios.de} or
printed by request.

\bibitem{berlios} Berlios.de project management site for HALT. \href{http://developer.berlios.de/projects/halttool}{http://developer.berlios.de/projects/halttool}

\bibitem{home} HALT home page. \href{http://halttool.berlios.de}{http://halttool.berlios.de}

\end{thebibliography}

\end{document}
